\chapter{Dictionnaire de données}


\begin{flushleft}

\small

\begin{tabular}{|p{1.0in}|p{2.8in}|p{0.9in}|p{1.4in}|} \hline 
\textbf{Action} & \textbf{Définition} & \textbf{Traduction} & \textbf{Nom Informatique} \\ \hline  \hline 
Collecte d'informations & Établir une liste de faits qui justifie la réalisation d'une tâche & Opération & Data Collect \\ \hline 
Traitement des informations & A partir des faits que l'on a listés, on établit une liste de tâches pertinentes à réaliser. & Opération & Data Process \\ \hline 
Organisation des données & Organisation des données selon certains critères:\newline  Le contexte, l'environnement~dans lequel on se situe, les outils disponibles\newline  Temps disponible, Le temps alloué avant la réalisation de la prochaine tâche\newline  Capacité physique, les ressources dont on dispose pour réaliser la tâche\newline  Priorité~: Dans le cas où des tâches se confondent sur les critères précédents, on définit une hiérarchie entre elles pour savoir lesquelles sont les plus importantes\newline  & Opération~ & Data Organisation \\ \hline 
Mise au point & Réactualisation de l'ensemble du système de façon périodique ou ponctuelle. Mise à jour du contexte, des listes de tâches, etc... & Opération & Review \\ \hline 
\textbf{Notion} & \textbf{Définition} & \textbf{Traduction} & \textbf{Nom Informatique} \\ \hline  \hline 
Tâche & Tâche, action à réaliser & classe & Task \\ \hline 
Projet & Un projet est un conteneur de tâches et de sous projets & classe & Project \\ \hline 
Contexte & L' environnement dans lequel la tâche doit être réalisée, ainsi que les outils nécessaires à sa réalisation & classe abstraite & Context \\ \hline 
\end{tabular}




\normalsize


\end{flushleft}




