\section{Composant ProcessManager}

Le processManager est le composant dédié au traitement métier de
l'application.Il fait parti du WebServer et intervient une fois que le
 ConnectionManager a établit la connexion à un serveur GTD et/ou ToodleDo.
Son rôle est donc d'effectuer les calculs et vérification. Ses fonctionnalités
sont les suivantes :\\
\begin{itemize}
  \item CRUD Idées
  \item CRUD Tâches
  \item CRUD Projet
  \item CRUD Notes
  \item CRUD Contexte
  \item Vérification des contraintes métier avant chaque opération
  \item Synchronisation des données
\end{itemize}


\subsection*{Couche métier de l'application}
Le ProcessManager contient toute la couche métier de l'application. A un niveau
plus générique, elle correspond au données stockées dans le serveur GTD. Ces
classes seront cependant amenées à 'voyager' entre les composants car elles
seront utilisées comme conteneur de données dans les échanges.

    
\subsection*{Précondition à chaque opération}
Avant chaque opération demandée à la couche métier, il est nécessaire de
vérifier que les modifications pourront être faites sur les serveurs distants.
Pour ce faire, le ProcessManager peut vérifier que la session est utilisable
via l'interface qui la relie au ConnectionManager, à savoir IConnManager.

\subsection*{Execution d'une opération}
Le processManager est le coeur métier de l'application, il se situe donc entre
l'ihm et les données. Il reçoit donc les appels du WebClient, modifie sa
représentation interne des donneés puis demande la modification au DataManager.
Cette chaine d'execution est la même pour chaque opération. En effet,
l'application travaille directement avec les serveur GTD et ToodleDo, il n'est
en effet pas concevable de fournir une application mise à disposition sur un
serveur Web, sans internet. Le mode non connecté n'est donc pas recevable; la
couche données de l'application est donc le serveur GTD ou ToodleDo lui même.

\subsection*{Synchronisation des données}
Le composant processManager permet également d'assurer la synchronisation entre les serveurs GTD et ToodleDo. C'est lui qui prend en charge la mise à jour des données d'un serveur vers l'autre. Pour cela, il va notamment utiliser les dates de modification des entités.
