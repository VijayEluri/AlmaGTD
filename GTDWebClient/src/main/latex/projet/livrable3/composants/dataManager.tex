
\section{Composant DataManager}


Le composant DataManager va faire le lien entre la couche métier (le composant
ProcessManager) et les serveurs GTD / ToodleDo. Il va donc permettre de répondre
au problématiques suivantes : \\

\begin{itemize}
 \item Permettre la communication entre le ConnectionManager et les serveurs GTD
 et ToodleDo
 \item Permettre de renvoyer à la couche métier les données en provenance du
 serveur GTD et ToodleDo
 \item Permettre la synchronisation du serveur GTD au serveur ToodleDo
 \item Permettre la synchronisation du serveur ToodleDo au serveur GTD\\
\end{itemize}




\subsection{Gestion des serveurs GTD et ToodleDo}

Le composant DataManager doit rendre l'utilisation des
serveurs ToodleDo et GTD transparente pour la couche métier (ProcessManager).
Ainsi, les données envoyées et reçues par celle-ci seront génériques et donc de
niveau métier. Cependant, lors de la transmission des données du DataManager au
serveur GTD et ToodleDo, le DataManager va devoir les adapter au bon format des
serveurs. En effet, un certain nombre de concepts diffèrent d'un serveur à l'autre. 
Par exemple, la notion de notes présente dans le serveur GTD ne l'est pas dans ToodleDo. 
De plus, il faut considérer différents scénarios possibles en cas de problèmes
avec les serveurs :


\subsubsection*{Gestion des connexions}

Comme le montre le diagramme de composant, le DataManager est le seul composant
relié aux serveurs. Il va donc devoir gérer les connexions et les transactions et va faire le lien entre la session utilisateur et les deux serveurs. Les couples login/password associés aux serveurs ne sont pas stockés dans ce composant mais dans le composant ConnectionManager.



\subsubsection*{Le serveur ToodleDo n'est plus disponible}

Si jamais le serveur ToodleDo venait à être indisponible, le DataManager
doit prévenir le composant ProcessManager, qui lui même informera le GUI.
Etant donné que ToodleDo ne dispose pas de toutes les fonctionnalités de GTD,
l'utilisateur ne constatera pas de différence sur la GUI. Une synchronisation
ultérieure sera nécessaire pour conserver une cohérence des données entre le
serveur GTD et ToodleDo.

\subsubsection*{Le serveur GTD n'est plus disponible}
Les fonctionnalités GTD étant plus nombreuses, en cas d'indisponibilité du
serveur GTD, l'affichage des données sera limitée sur la GUI. En effet, les
données, bien que génériques, seront limitées aux fonctionnalités du serveur
ToodleDo. En cas d'indisponibilité des deux serveurs, le Data manager
l'informera au ConnectionManager.
